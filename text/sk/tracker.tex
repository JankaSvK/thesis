\chapter{Tracker}

We considerate tracker as an algorithm used to detect a position of the object in
an image. We will present few tested trackers and their results in this task.
Firstly we provide a short description of simple straightforward tracker and
then we will describe more complicated trackers.

We use a word tracker generally for an algorithm able to detect object. Some
trackers but do not take an advantage of past images and information gained
from them. They simply detect an object in the each image. On the other hand
many good trackers using also information from previous images exist. For this
task some trackers from both categories.

%%%%%%%%%%%%%%%%%%%%%%%%%%%%%%%%%%%%%%%%%%%%%

\section{Simple Background Tracker}

This tracker takes a photo of the background at the beginning and calls it
\emph{pattern}. In order to detect an object in \emph{image} is taken a
comparison of the \emph{image} and \emph{pattern}. A Comparison is done by
taking a sum of an absolute difference for each colour (Red, Green, Blue) in
the images. 

As result, we get a map, where higher values mean bigger difference between the
colors of the \emph{pattern} and \emph{image} at given point. We will assume it
is caused by an object in front of the camera at given point.  As the next step
we will binarize the map with a given \emph{treshold}. At this point we will
find a countour with biggest area using OpenCV library. Centerpoint of the
rectangle of this contour will be estimated position of our object in the
image..

\todo[inline]{Pomôže práve popísaný postup zapísať v niekoľkých riadkoch pseudokódu? Bolo
by to prehľadnejšie (na jedno pozretie jasné, bez čítania odstavca, ak čitateľ
vie, čo očakávať)}

\todo[inline]{Fotka vzoru (farebne), fotka s objektom, absolute\_diff, výsledok po rôznych tresholdoch}

\todo[inline]{Nájdenie contúr na obrázku, zobrazenie najväčšej, bod ako stred}

\subsubsection{Advantages}
\begin{itemize}
\item Quite straight-forward implementation
\item Ability to recognise variate object without having specific color or pattern.
\end{itemize}

\subsubsection{Disadvantages}
\begin{itemize}
\item Cannot recover from even small movement of the camera.
\item Moving object with a hand will cause recognizing the hand also as an object and will result wrongly estimated center of the moving object. Same problems cause shadows.
\end{itemize}

%%%%%%%%%%%%%%%%%%%%%%%%%%%%%%%%%%%%%%%%%%%%%

\section{Adaptive Background Tracker}

In order to make our background tracker robust to small movement of camera we
are going to update our \emph{pattern}. This could be simply done by defining \emph{pattern}
as the mean of last \emph{n} images.

\subsubsection{Advantages}
\begin{itemize}
\item Robust against movements of the camera
\end{itemize}

\subsubsection{Disadvantages}
\begin{itemize}
\item If the object is not moving it will disappear as becoming a part of the
background. After moving it will recover properly.
\end{itemize}

%%%%%%%%%%%%%%%%%%%%%%%%%%%%%%%%%%%%%%%%%%%%%

\section{More advanced trackers}

%%%%%%%%%%%%%%%%%%%%%%%%%%%%%%%%%%%%%%%%%%%%%

\section{Comparison of trackers}

\subsubsection{Experiment with simplyfied environment}

Given a video sequence XYZ second long we studied an accuracy of trackers. The background is one colored and moving object is a red circle.

\todo[inline]{ Robot sledujúci čiaru do štvorca s červeným kruhom z vrchu (kamery sa dívajú zvrchu). Porovnanie bude uvedené ako čiary, ktoré sú nakreslené pomocou zachytených bodov.}

\subsubsection{Experiment in complex environment}

Background consistsed of many colors and patterns. Moving object has a pattern and is partially colored as the background.

\todo[inline]{ Vyskytli sa veci, ako, že stratil polohu? V akom percente? Mám ako rozumne vyjadriť presnosť tých súradníc lepšie než od oka?}

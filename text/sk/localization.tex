\chapter{Localization}

In previous few chapters we covered steps to obtain position of the object in
2D images. In this chapter we will take closer look of obtaining postition of
the object in 3D by combining iformation from more cameras.

At this point we have computed not only intristic matrices of the cameras but
also the rotation matrix and translation vector (from mono camera calibration
and stereo calibration). Our goal is to from tuple of coordinates from the
images taken by cameras to get a position of the object in world coodinates
(3D).

\section{Projection matrices}
We define projection matrix as tranformation matrix P, where stands: $x = P
\dot X$, where $X$ denote a vector of size 4x1 - homogenous world coordinates
of the object and $x$ denotes homogenous object coordinates in the image plane
of the camera - a vector 3x1.

\subsection{World coordinate system}
Firstly we will compute projection matrices. We define our world coordinate
system as orthogonal, with the origin in the center of the first (usually left)
camera. The positive part z-axis is poiting in front of the camera and and
below the camera is positive y-axe and to the right is positive x-axe.
\todo[inline]{Obrazok suradnicoveho systemu}

\subsection{Computing projection matrices}
Projection matrix can be decomposed as $P = K[R|T]$, where K is intristic
camera matrix and $[R|T]$ is extrinstic matrix. $R$ is a rotation matrix and
$T$ translation vector. For the first camere we get this equation: $ P_1 = K_1
\dot [I_3 | 0_3]$. $K_1$ first camera instrictic parameters matrix, $I_3$
denotes identity matrix 3x3 and $0_3$ zero vector.

For the second camera results from stereo calibration will be used and since
then we get this relation: $P_2 = K_2 \dot [R | T]$, where $K_2$ is second
camera instrictic parameters matrix, $R$ rotation vector and $T$ translation
vector obtained by stereo calibration.

More about the decomposition itself could be found in article by \citet{computervisionblog}.
\todo[inline]{V programe aktualne nerobim s distortion coeffs, pretoze uz aj bez toho su rozumne vysledky}

\section{Triangulation}
Now when we know projection matrices we can formalize our problem as 
\begin{equation}
x_1 = P_1X, x_2 = P_2X \label{projection-statements}
\end{equation}
with the goal to find $X$. Since errors may occure during
measurement of $x_1$, $x_2$ and calibration. In further steps we consider that
calibration results are provided with high accurancy compared to measurement of
$x_1$ and $x_2$ (that is the reason to have longer calibration with more images
once).

\subsection{Simple linear triangulation}
We are going to shortly describe how is triangulation working under the hood.

The results of cross product of vector itself is null vector. Since
\ref{projection-statements} we can write our crossproduct for a specific point
as $x \times (PX) = 0$. We denote point $x = (x, y, w)$, where $w = 1$ since
these coordinates are homogenous -- in other words up to scale factor $w$.

We can then rewrite $x \times (PX) = 0$ in following way:

$$ y(p^{3T}X) - w(p^{1T}X) = 0 $$
$$ w(p^{3T}X) - x(p^{2T}X) = 0 $$
$$ x(p^{2T}X) - y(p^{1T}X) = 0 $$

Where $p^{iT}$ denotes ith row of $P$. Since $w = 1$ we can equally write:

$$ x(p^{3T}X) - (p^{1T}X) = 0 $$
$$ y(p^{3T}X) - (p^{2T}X) = 0 $$
$$ x(p^{2T}X) - y(p^{1T}X) = 0 $$

These equations are linear in the components of X. Only two equations are
linear independent, since the third one could be obtained as from the sum $y$
times first row and $-x$ times second row.

Therefore an equation of form $AX = 0$ can then be composed using two points $x_1 = (x, y, 1)$ and $x = (m, n, 1)$:

\[
A = \begin{pmatrix}
x(p_1^{3T}X) - (p_1^{1T}X) \\
y(p_1^{3T}X) - (p_1^{2T}X) \\
m(p_2^{3T}X) - (p_2^{1T}X) \\
n(p_2^{3T}X) - (p_2^{2T}X) \\
\end{pmatrix}
\] 

For each image two equations were included, giving a total of four equations in
four homogenoeours unknowns.

Without an error during measurements a point $X$ satisfiying $AX = 0$ would
exists. Due to errors it might not exists. As a next step Homogenous method
(DLT) is used to find a solution. More about the method could be found in \citet*{multiple-view-geometry}.

\section{Implementation note}
For triangulation we used OpenCV function triangulatePoints($P_1$, $P_2$,
$x_1$, $x_2$), which is based on simple triangulation method with use of DLT
method for solving equations.

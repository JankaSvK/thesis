\chapter{User documentation}
This part of the documentation is focused on the end user. We introduce
installation details and manual for program usage.

\section{Installation guide}
This section documents the process of downloading til running the program.

\subsection{Dependencies}
Following packages are required to compile the application. We also provide
versions of packages used to create and test our implementation.

\begin{center}
\begin{tabular}{l l}
	package	&	version \\ \hline
	Python	&	3.3 \\
	NumPy	&	x.x \\
	OpenCV	&	x.x 
\end{tabular}
\end{center}

In case OpenCVcontrib could not be obtained, the program can be used with
OpenCV (version x.x). As the result only detection trackers will be available.

\subsection{Hardware requirements}
The software was primarly tested on a system with Intel(R) Core(TM) i5-7300HQ
CPU (2.50GHz, 2496 Mhz, 4Core), 16GB RAM running Microsoft Windows 10
Enterprise. Minimal requirements are lower, but the copmutation power reflects
on frequency of getting localization results.

Also two cameras are needed. We tested on .... and .... . Laptop camera may be
used. Requirements for the cameras are 640x320 px and 20 FPS.

\subsection{Downloading the application}
The application can be downloaded from \url{https://github.com/JankaSvK/thesis}.

\section{Usage guide}

\subsection{Running the application}
In the folder \verb+application+ we find an entry point for our application
\verb+Main.py+.

Different options may be passed to the program (ref to list). In case no option
is passed program runs on first two available cameras. Firstly calibration for
each camera is done and then stereo calibration. As a tracker is used \verb+KCF+.

\begin{code}
Usage: Main.py [options]

Options:
  -h, --help            show this help message and exit
  --calibrationi\_results1=CALIB1
                        Calibration results for the first camera
  --calibration\_results2=CALIB2
                        Calibration results for the second camera
  --stereo\_calibration\_results=STEREO
                        Stereo calibration results
  --video\_recording1=VIDEO1
                        Video recording for the first camera
  --video\_recording2=VIDEO2
                        Video recording for the second camera
  --tracker=TRACKER     The algorithm used for tracking
\end{code}
% len vykopiruj z toho co ti da -h

\subsection{Notes for options}
Video file
- formats \verb+TODO+ are accepted.

As \verb+TRACKER+ may be used a name of implemented trackers. We introduce a list of the names \verb+KCF, SIMPLEBACKGROUND, PATTERNMATCHING, HSV, TLD, ...+.

For calibration results JSON in this format are used:
Calibration results:


Stereo calibration:
\begin{code}

\end{code}
\subsection{Reusing calibration results}
Calibration results from previous runs may be reused by adding the option for specific camera. After succesfull calibration the results are automatically saved in this structure:

\begin{code}
calib\_results/
 - 1/
 - 2/
 - stereo\_calib\_results/
\end{code} 	

The file for specific calibration result is named in this manner:
{year}-{month}-{day}-at-{hour}-{minute}.json when the calibration successfully
ended.

\subsection {Inspecting localization data}
After appropiate close of the program, localization data are automatically
saved in \verb+localization\_data/+. Same convention for file names is used as
with calibration results.

Sample of localization data exaplained:
TODO

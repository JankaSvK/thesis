\chapter{Experiments} 

For the designed system we considered multiple experiments to
verify its accuracy. Experiments in this part focus on the localization process
and overall experiments. The trackers experiments are located in the chapter
Tracker.

\section{Calibration and localization}

We decided to exclude errors caused by trackers and firstly test the system
without them. Therefore, we propose an experiment, which will set fixed points
in camera views. Then we estimate position in 3D for these fixed points.

It is complicated to measure distance from the origin of the coordinate system
to the point (and also find the "real" coordinates with this origin).
Therefore, we compute a distances between these points in the real world and
between the estimated points. Then we compare the results.
 
To skip the tracking part, we created a new tracker. This new tracker always
return the coordinates on which it was initialized. Doing this, we excluded
tracker from our process.

We choosed a grid (view figure \ref{fig:grid}) as our pattern for experiments.
The vertical lines are circa 400~mm long and the distance between them is circa
200~mm. We measured distances between the crossings. We numbered the crossing,
the left column is from top to the bottom from 1 to 7 and the right column from
8 to 14 (see the figure \ref{fig:ladder_numbered}).

The setup of the cameras was nearly parralel and the distance between them was
circa 16~cm. Selected points are displayed in the figure \ref{fig:ladder_ground}. The
results from this experiment are listed in the table \ref{table:distances}. 

From the results we can see that results for horizontal lines (with the length
400~mm, third part of the table) are very accurate. On the other hand, the
results for the vertical lines were quite unstable. For the left column it
seems, that results get worse when, the points are further away from the
camera. But the results from the right column bring the hypothesis into
question. 

\begin{table}
\centering
\begin{tabular}{|r|r|r|r|r|}
\hline
From	& To	& Real length (mm) & Computed length(mm) & Error (\%) \\
\hline
\hline
\input{distances.txt}
\hline
\end{tabular}
\caption{Results of the experiment focused on the distances}
\label{table:distances}
\end{table}

We were interested, if the results are good not only on the one plane -- in our
case the ground -- but also in the others. Now we have tested the precision in
two axis, which were perpendicular to each other (vertical and horizontal
lines). We have noticed, that the results for the vertical lines are less
precise. Therefore, we decided to test the last ax and its precision. We places
a wall with the marks, which is perpendicular to the ground and measured the
distance between the marks. The setup can be seen in the figure
\ref{fig:table}. The marks are numbered as displayed in the image
\ref{fig:tablenumbered}. The results are included in the table
\ref{table:distances}. 

We did same experiment with another setup, when the cameras were 63~cm away
from each other. We were interested, if the results are better in parallel
setup, or not. From the results the is no evidence of the different precision
and the differences may be still cause by different precision of the
calibration. The results for this second experiment are listed in the appendix
in the table \ref{table:distances-second}. Same numbering of the marks was used.

\begin{figure}
\centering
\includegraphics[width=0.8\linewidth]{img/experiments/right-ladder-numbered.png}
\caption{Numbering the points}
\label{fig:ladder_numbered}
\end{figure}

\begin{figure}
\centering
\begin{subfigure}{0.48\linewidth}
	\includegraphics[width=\linewidth]{img/experiments/left-ladder.png}
\end{subfigure}
\begin{subfigure}{0.48\linewidth}
	\includegraphics[width=\linewidth]{img/experiments/right-ladder.png}
\end{subfigure}
\caption{Selecting the points from the videos}
\label{fig:ladder_ground}
\end{figure}



\begin{figure}
\centering
\begin{subfigure}{0.48\linewidth}
	\includegraphics[width=\linewidth]{img/experiments/table-left.jpg}
	\caption{Left view of the camera with marks}
	\label{fig:tablenumbered}
\end{subfigure}
\begin{subfigure}{0.48\linewidth}
	\includegraphics[width=\linewidth]{img/experiments/table-right.jpg}
	\caption{Right view of the camera}
\end{subfigure}
\caption{The vies of the camera on the penpedicular plane to the ground}
\label{fig:table}
\end{figure}

\begin{figure}
\centering
\begin{tikzpicture}[]
	\draw (0, 0) -- (12, 0);
	\draw (0, 4) -- (12, 4);	
	\draw (0, 0) -- (0, 4);
	\draw (2, 0) -- (2, 4);
	\draw (4, 0) -- (4, 4);
	\draw (6, 0) -- (6, 4);
	\draw (8, 0) -- (8, 4);
	\draw (10, 0) -- (10, 4);
	\draw (12, 0) -- (12, 4);
	\node [right] at (0.1, -0.25) {200mm};
	\node [right] at (2.1, -0.25) {200mm};
	\node [right] at (4.1, -0.25) {200mm};
	\node [right] at (6.1, -0.25) {200mm};
	\node [right] at (8.1, -0.25) {200mm};
	\node [right] at (10.1, -0.25) {200mm};
	\node [right] at (12.1, 2) {400mm};
\end{tikzpicture}
\caption{Pattern used for experiments}
\label{fig:grid}
\end{figure}

\section{Overall experiments}

As the last step, we decided to test whole system in real environment. This
time we included all parts of the application -- calibration, tracking and also
localization. In the previous experiments, we tested the liability of the
trackers and the precision of the results. Now we test the usability of the
whole system as the whole.

\subsection{Experiment with the one small object}

The most important experiment -- the experiment which decides if the system is
usable -- is experiment under real conditions. We created for our autonomous
robot a square to follow, with the lenfth of the side equal 14.5~cm (see figure
\ref{fig:robot-square}). The robot did four laps around the square. The
results are displayed in the figure \ref{fig:square-results}.

We can see on the left picture, that the drawn line is similar to our shape.
Not having sharp edges, since our robot make a small ars. On the other hand,
the results from the another view are not so amazing. the coordinates have a
lot of noise over Z-ax. The noise is only present, if the robot is following
horizontal lines of the square. As the last step we computed maximum of the
distances between any two points, which were located. In this case, the ideal
value would be the length of the diagonal, which is equal to 20.5061~cm. Our
results are between 18.54~cm to 19.38~cm, depending on the run. That means,
that the error is only usually between 1-2~cm..

\begin{figure}
\includegraphics[width=\linewidth]{img/experiments/square-robot.png}
\caption{Experiment with the robot}
\label{fig:robot-square}
\end{figure}

\begin{figure}
\centering
\begin{subfigure}{0.48\linewidth}
	\includegraphics[width=\linewidth]{img/experiments/square-nice.png}
\end{subfigure}
\begin{subfigure}{0.48\linewidth}
	\includegraphics[width=\linewidth]{img/experiments/square-ugly.png}
\end{subfigure}
\caption{Results of the localization -- four laps by the robot}
\label{fig:square-results}
\end{figure}

\subsection{Experiment with two objects}

This experiment will test the ability of the system to track multiple objects.
We decided to choose an environment where Simple background nor HSV tracker can
be used (since if they can be used they perform very well). The rest of the
trackers are mostly slower, more problematic.

We can see the setup for the experiment in the figure \ref{fig:two-init}. We
can see two boxes, both in shades of blue. During testing many trackers failed
to track and lost the object. We choosed a \corr{} tracker. It is medium
fast tracker, but when tracking two object -- means 4 \corr{} trackers --
it too slow. Therefore it sometimes miss the images, which passed while it was
computing. Hence, sometimes an interaction of the user is needed, to
reinitialize a tracker.

Sample results of the trajectories for two objects is displayed in the figure
\ref{fig:two-trajectories}. During the experiment, the boxes were moving
between the yellow marks and swapped place. We can see these four marks also in
both trajectories quite close to each other. 

\begin{figure}[p]
\includegraphics[width=\linewidth]{img/experiments/two-objects.png}
\caption{Initialization of the two objects}
\label{fig:two-init}
\end{figure}

\begin{figure}
\centering
\begin{subfigure}{0.48\linewidth}
	\includegraphics[width=\linewidth]{img/experiments/trajectories1.png}
\end{subfigure}
\begin{subfigure}{0.48\linewidth}
	\includegraphics[width=\linewidth]{img/experiments/trajectories2.png}
\end{subfigure}
\caption{Sample trajectories from tracking two objects}
\label{fig:two-trajectories}
\end{figure}

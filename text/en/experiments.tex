\chapter{Experiments} 

%For the designed system we considered multiple experiments to
%verify its accuracy..
%
%In further text, we will mention an autonomous robot able to follow a black
%line.  It was used for our experiments but could be replaced by any other
%object. The advantage of an autonomous robot is repetitive of its movements and
%also no other distractions in the scene except the moving object. The robot has
%a cylindrical shape with 5~cm in diameter and height of 2~cm.
%
%%\section{Trackers comparison} 
%%
%%We tested all trackers in two situations. Since
%%all used trackers were computability admissible, their complexity will not be
%%examined. The project goal is to track a moving object. Therefore we consider
%%these qualities: stability in a fast moving object, recovery from full
%%...zmiznutia z obrazovky..., precision.
%%
%%Since some trackers use feature points, we provide two testing scenarios. First
%%is a common object to track -- red circle. The second experiment consists of
%%tracking robot - sample usage of the designed system. Both are evaluated on a white
%%background and also on noisy background.
%
%%\section{Localization precision}
%
%\section{Distances}
%
%The results of localization depends on two factors: calibration and tracking.
%We decided to firstly estimate errors caused by calibration process.
%
%
%
%\section{Distances on the square}
%
%Firstly, we were interested how accurate is localization. In order to ommit a
%tracker errors, we prepared a static setup.  We placed four marks on the
%ground. Both cameras had all four points in the view. The marks formed a
%square, with a length of 30~cm per side.
%
%Then we runned our programme for each point to get an estimated position. We
%computed a norm between each two points and we display results in the figure
%\ref{fig:squre-distances}.
%
%\section{Distances on the cube}
%As a next step, we were interested if the precision is the same also using
%non-planar object. We choosed a cube as our object.
%
%\subsection{Hausdorff distance} To tell how close or far are two
%curves apart we used Hausdorff distance.
%
%\subsection{Square experiment} We used the robot to follow a black line in the shape of
%square.
%
%\todo[inline]{Graf zavislosti reprojection error a chyby merania}
%\todo[inline]{Graf zavislosti zavislost normy translation vector a chyby
%merania}

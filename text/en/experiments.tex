\chapter{Experiments} For the designed system we considered multiple experiments to
verify its stability and precision.

In further text, we will mention an autonomous robot able to follow a black
line.  It was used for our experiments but could be replaced by any other
object. The advantage of an autonomous robot is repetitive of its movements and
also no other distractions in the scene except the moving object. The robot has
a cylindrical shape with 5~cm in diameter and height of 2~cm.

\section{Trackers comparison} We tested all trackers in two situations. Since
all used trackers were computability admissible, their complexity will not be
examined. The project goal is to track a moving object. Therefore we consider
these qualities: stability in a fast moving object, recovery from full
...zmiznutia z obrazovky..., precision.

Since some trackers use feature points, we provide two testing scenarios. First
is a common object to track -- red circle. The second experiment consists of
tracking robot - sample usage of the designed system. Both are evaluated on a white
background and also on noisy background.

\section{Localization precision} After evaluating experiments on tracker a
further step was to test the precision of the overall system.

\subsection{Hausdorff distance} To tell how close or far are two
curves apart we used Hausdorff distance.

\subsection{Square experiment} We used the robot to follow a black line in the shape of
square.

\todo[inline]{Graf zavislosti reprojection error a chyby merania}
\todo[inline]{Graf zavislosti zavislost normy translation vector a chyby
merania}

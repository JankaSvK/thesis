\chapter{Experiments} For designed system we considered multiple experiments to
verify its stability and precision.

In further text we will mention an autonomous robot able to follow black line.
It was used for our experiments, but could be replaced by any other object. The
advantage of autonomous robot is repetitivity of its movements and also no
other distractions in the scene except the moving object. The robot has a
cylindrical shape with 5cm in diameter and height od 2 cm.

\section{Trackers comparison} We tested all trackers in two situations. Since
all used trackers were computabily admissible, their complexity will not be
examined. Project goal is to track moving object therefore we consider these
qualities: stability in fast moving object, recovery from full ...zmiznutia z
obrazovky..., precision.

Since some trackers use feature points we provide two testing scenarios. First
is typical object to track -- red circle. Second experiment consists of
tracking robot - sample usage of designed system. Both are evaluated on white
background and also on noisy background.

\section{Localization precision} After evaluating experiments on tracker our
further step was to test precision of the overall system.

\subsection{Hausdorff distance} In order to tell how close or far are two
curves apart we used Hausdorff distance.

\subsection{Square experiment} We used robot to follow a black line in shape of
square. 

\todo[inline]{Graf zavislosti reprojection error a chyby merania}
\todo[inline]{Graf zavislosti zavislost normy translation vector a chyby
merania}

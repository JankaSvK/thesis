\chapter{Related works} The subject of retrieving an object position in 3D is
well known and ... in many papers. Usage could be found in many different ...,
for example in the sport in controversial cases, in the robotics for better
navigation, but also in virtual reality for more impressive effect.

\section{Stereo vision systems}

In the paper \citet*{zheng2010study}, the authors research stereo vision. The
proposed system has a requirement of the parallel alignment of the cameras to
each other but also parallel to the floor. The authors test the accuracy of the
system and also the disparity of the results in different distances. They
provide several experiments measuring these disparities moving an object in an
environment with one colored background.

The paper \citet*{black2002multi} tries to solve a problem of the prediction
the object position under full occlusion. On the other hand, the authors of the
\citet*{yonemoto1998tracking} solve a problem of correct localization of the
object consisting of the multiple parts. Such objects often fall apart when
trying to include them in virtual reality. Proposed system estimated parameter
and fixed points for each part independently. Then the system uses
correspondence of the points within the movements.

The system of the object tracking and localization in the garage was introduced
in the paper \citet*{ibisch2915arbitrary}. The proposed system is connected to
few cameras sharing a part of their view. For object detection, they use a
method of the background subtraction. Since the system's primary use is in the
garage, they also propose a method to cope with the change of the lighting
caused by car lights. Their primary goal is to predict possible car collisions
and therefore provide a data for a warning system.

\section{The use of the systems in the sports}

An excellent example of the usage of such a system is to solve controversial
cases in the sports. One of the most famous systems is Hawk Eye (shortly
described in \citet*{owens2003hawk}). The system provides access to the
trajectory of the ball and can replay it to referees. This system needs an
expensive setup equipped with high-speed cameras. The software itself is
expensive.

\section{The use of the systems in the robotics}

Described principles have a great usage in robotics, for better navigation,
object manipulating and so on. For example, for the robotic competition
RoboCup, category Soccer many systems for ball tracking were developed. Robots
competing in this category are usually equipped with an omnidirectional camera.
This camera is pointing vertically up, where the light reflects against the
mirror. This principle provides a full 360\degree image of the environment
provided by only one camera. In this case, it is possible to track an object by
one camera, since additional information about the ball (such as color and
size) are provided in advance. On the other hand, such a system is not static,
the background changes with every movement of the robot. For better precision,
another camera may be used. For more information about it, we refer to the
paper \citet*{kappeler20103d}. 

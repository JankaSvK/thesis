\chapter{Related works} 

The subject of retrieving an object position in 3D is well known and discussed
in many papers. Usage could be found in many different areas, for example in
sport in hard to decide cases, in the robotics for better navigation, but
also in virtual reality for more impressive effect.

\section{Stereo vision systems}

In the paper \citet*{zheng2010study}, the authors research stereo vision. The
proposed system has a requirement of the parallel alignment of the cameras to
each other but also parallel to the floor. The authors test the accuracy of the
system and also the disparity of the results in different distances. They
provide several experiments measuring these disparities moving an object in an
environment with one colored background.

The paper \citet*{black2002multi} tries to solve a problem of the prediction
the object position under full occlusion. The authors of the
\citet*{yonemoto1998tracking} solve a problem of correct localization of the
object consisting of the multiple parts. Such objects often fall apart when
trying to include them in virtual reality (for example a snowman can be mapped
as three balls apart of each other). Proposed system estimated parameters and
fixed points for each part independently. Then the system uses correspondence
of the points within the movements.

The system of the object tracking and localization in the garage was introduced
in the paper \citet*{ibisch2015arbitrary}. The proposed system uses few cameras
that share a part of their view. For object detection, they use a method of the
background subtraction. Since the system's primary use is in the garage, they
also propose a method to cope with the change of the lighting caused by car
lights. Their primary goal is to predict possible car collisions and therefore
provide a data for a warning system.

\section{Use of the systems in sports}

An excellent example of an usage of such a system is to solve cases in sports,
where sight of the naked human eye is inconclusive. Over the last few years,
one of the most famous systems is Hawk Eye (shortly described in
\citet*{owens2003hawk}). The system provides access to the trajectory of the
ball and can replay it to referees. Furthermore, event most popular soccer
competitions are experimenting with system for detecting if the ball cross the
goal line or not.  This system needs an expensive setup equipped with
high-speed cameras. The software itself is of high price.

\section{Use of the systems in robotics}

Described principles have a great usage in robotics, for better navigation,
object manipulating and so on. For example, for the robotic competition
RoboCup, category Soccer many systems for ball tracking were developed. Robots
competing in this category are usually equipped with an omnidirectional camera.
This camera is pointing vertically up, where the image reflects in a mirror.
This principle provides a full 360\degree image of the environment provided by
only one camera. In this case, it is possible to track an object by single
camera, since additional information about the ball (such as color and size)
are provided in advance. On the other hand, such a system is not static, the
background changes with every movement of the robot. Nevertheless, usage of
additional camera may significatly improve the precision.. For more information
about it, we refer to the paper \citet*{kappeler20103d}. 

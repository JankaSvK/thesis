\chapter{Related works} 

The subject of retrieving an object position in 3D is well known and discussed
in many papers. Applications can be found in many different areas, for example in
sports in unclear situation, in the robotics for better navigation and
also in augmented reality for more impressive effect.

\section{Stereo vision systems}

In the paper by \citet*{zheng2010study}, the authors research stereo vision. The
proposed system has a requirement of the parallel alignment of cameras both to
each other and also parallel to the floor. The authors test the accuracy of the
system and also the disparity of the results in different distances. They
provide several experiments measuring these disparities moving an object in an
environment with one-color background.

The paper by \citet*{black2002multi} tries to solve a problem of the prediction
the object position under full occlusion. The authors of the
\citet*{yonemoto1998tracking} solve a problem of correct localization of the
object consisting of multiple parts. These parts of the object often fall apart
when trying to include them in virtual reality (for example a snowman can be
mapped as three balls apart of each other). The proposed system estimated
parameters and fixed points for each part independently. Then the system uses
correspondence of the points within the movements.

The system for the object tracking and localization in the garage was introduced
in the paper \citet*{ibisch2015arbitrary}. The proposed system uses a few cameras
that share parts of their views. For object detection, they use a method of the
background subtraction. Since the system's primary use is in the garage, they
also propose a method to cope with the change of lighting caused by car
lights. Their primary goal is to predict possible car collisions and therefore
to provide a data for a warning system.

\section{Use of the systems in sports}

An excellent example of an usage of a system for tracking an object in 3D space
is to solve unclear situations in sports, where sight of the naked human
eye is inconclusive. Over the last few years, one of the most famous systems is
Hawk Eye (shortly described in \citet*{owens2003hawk}). The system provides
access to the trajectory of the ball and can replay it to referees.
Furthermore, even the most popular soccer competitions are experimenting with
the system for detecting if the ball crosses the goal line or not.  This system needs
an expensive setup equipped with high-speed cameras and the software itself is of
high price.

\section{Use of the systems in robotics}

The principles described in previous sections have an extensive usage in
robotics, for better navigation, object manipulating and so on. For example, in
the robotic competition RoboCup, competitors of the Soccer category developed
many systems for ball tracking. Robots competing in this category are usually
equipped with an omnidirectional camera.  This camera is pointing vertically
up, where the image reflects in a mirror.  This principle provides a full
360\degree image of the environment provided by only one camera. In this case,
it is possible to track an object by single camera, since additional
information about the ball (such as color and size) is provided in advance. On
the other hand, such a system is not static, the background changes with every
movement of the robot. Nevertheless, usage of additional camera may
significantly improve the precision. For more information about it, we refer to
the paper \citet*{kappeler20103d}.


\section{Dove-eye}

The idea of this work is based on its predecessor, Dove-eye presented in the
paper by \citet*{dove-eye}. This project uses automatic calibration process to
get the information about cameras and provides several ways to track an
object. Localization results are displayed live.

\section{Our approach}

Starting with the ideas used in Dove-eye, we developed a new implementation for
this task. We improve the tracking process by providing many different
trackers, which gives an opportunity to choose the one which is best suitable for
given environment. We provide an easy way to correct the tracker by its
reinitialization. Furthermore, no knowledge of the internals is needed to add a
new tracker.  In difference to some previously mentioned papers, we do not
require special alignment of the cameras. 

We also implement an algorithm similar to the one mentioned by
\citet*{ibisch2015arbitrary} and others. As opposite to paper by
\citet*{kappeler20103d} we do not use any previous information about the object
tracked.

This thesis also considers a case of tracking multiple objects. It provides a
way, to initialize tracking objects. The comparison of the trackers is
provided, including their speed, accuracy, but also ability to track
multiple objects.

We conclude the thesis with several experiments, including ones with autonomous
robot, to test the system suitability for applications in robotics.

\todo[inline]{Chyba analyza z pohladu toho, co budete delat, co jste si z tech systemu vzala jako motivaci, v cem se hodi, ci nehodi pro vase zadani}

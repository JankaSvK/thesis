\chapter{Related works} 

The subject of retrieving an object position in 3D is well known and discussed
in many papers. Usage can be found in many different areas, for example in
sports in unclear situation, in the robotics for better navigation and
also in augumented reality for more impressive effect.

\todo[inline]{Mozno este napises o tom, ako potom mozes postavit postavicku na spravne miesto}

\section{Stereo vision systems}

In the paper by \citet*{zheng2010study}, the authors research stereo vision. The
proposed system has a requirement of the parallel alignment of cameras both to
each other and also parallel to the floor. The authors test the accuracy of the
system and also the disparity of the results in different distances. They
provide several experiments measuring these disparities moving an object in an
environment with one-color background.

The paper by \citet*{black2002multi} tries to solve a problem of the prediction
the object position under full occlusion. The authors of the
\citet*{yonemoto1998tracking} solve a problem of correct localization of the
object consisting of multiple parts. Such objects often fall apart when
trying to include them in virtual reality (for example a snowman can be mapped
as three balls apart of each other). The proposed system estimated parameters and
fixed points for each part independently. Then the system uses correspondence
of the points within the movements.

The system for the object tracking and localization in the garage was introduced
in the paper \citet*{ibisch2015arbitrary}. The proposed system uses a few cameras
that share parts of their views. For object detection, they use a method of the
background subtraction. Since the system's primary use is in the garage, they
also propose a method to cope with the change of lighting caused by car
lights. Their primary goal is to predict possible car collisions and therefore
to provide a data for a warning system.

\section{Use of the systems in sports}

An excellent example of an usage of a system for tracking an object in 3D space
is to solve unclear situations in sports, where sight of the naked human
eye is inconclusive. Over the last few years, one of the most famous systems is
Hawk Eye (shortly described in \citet*{owens2003hawk}). The system provides
access to the trajectory of the ball and can replay it to referees.
Furthermore, even the most popular soccer competitions are experimenting with
the system for detecting if the ball crosses the goal line or not.  This system needs
an expensive setup equipped with high-speed cameras an the software itself is of
high price.

\section{Use of the systems in robotics}

The principles described in previous sections have a extensive usage in
robotics, for better navigation, object manipulating and so on. For example,
for the robotic competition
RoboCup, the category Soccer many systems for ball tracking were developed. Robots
competing in this category are usually equipped with an omnidirectional camera.
This camera is pointing vertically up, where the image reflects in a mirror.
This principle provides a full 360\degree image of the environment provided by
only one camera. In this case, it is possible to track an object by single
camera, since additional information about the ball (such as color and size)
is provided in advance. On the other hand, such a system is not static, the
background changes with every movement of the robot. Nevertheless, usage of
additional camera may significatly improve the precision. For more information
about it, we refer to the paper \citet*{kappeler20103d}.

\section{Our approach}

After a review previously mentioned papers, we followed the idea of stereo
vision provided by two cameras. We decided to propose a system which does not
need any prior information about the object, but needs to be selected by the
user. Also, we decided to test several cameras setups, not requiring specific
one. Since, not many papers mention the tracking method which was used, we will
try several to find the best available option. Due to great potential of the
system in robotics, we will test our system also with a autonomous robot.

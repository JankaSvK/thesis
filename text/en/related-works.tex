%%% Súvisiace práce

\chapter{Súvisiace práce}

%%%\section{Úprava práce}

Problematike získavaniu polohy objektu v priestore sa venuje mnoho práci.
Tomuto rozšírenému záujmu vďačíme nzvýšenému záujmu o virtuálnu realitu ako aj
stále sa zvyšujúce požiadavky robotiky, či vyhľadávaním takejto pomoci pri
športe.

\section{Systémy priestorového videnia}

V článku \citet*{zheng2010study} sa autori zaoberajú otázkou priestorového
videnia. Jednou z podmienok systému je vzájomne rovnobežné postavenie kamier a
taktiež rovnobežné postavenie k podlahe. Okrem toho predpokláda taktiež, že
súradnice v priestore sú dobre zarovnané s optickými osami. Autori okrem
navrhnutého systému skúmali presnosť systému v hĺbke videnia a prezentovali
niekoľko experimentov v rôznych vzdialenostiach s dobre odlíšiteľným pozadím.

Experimenty s predikciou polohy objektu napriek jeho strate z obrazu sa venuje
článok \citet*{black2002multi}. Autori \citet*{yonemoto1998tracking} riešia
problém komplexných objektov pozostávajúcich z niekoľkých častí. Takéto objekty
sa častokrát pri mapovaní do virtuálnej reality rozpadajú. Navrhovaný systém
odhaduje parametry a body sledovania samostatne na každej časti, pričom
následne hľadá korešpondujuce dvojice.

Systém pre sledovanie a lokalizáciu objektov v garážiach bol navrhnutý v článku
\citet*{ibisch2015arbitrary}. Tento systém pozostáva z niekoľkých čiastočne sa
prekrývajúcich pohľadov kamier. K detekcii objektov využíva metódu s
porovnávaním pozadia.  Vzhľadom na to, že navrhovaný systém bol použitý do
garáži k predchádzaniu zrážok, tak je taktiež navrhnutá metóda k rozpoznávaniu
objektov rozšírená o redukciu šumu spôsobeného odrazom svetla. Tieto odrazy --
a teda presvetlené miesta spôsobujú hlavne svetlá vozidiel.

\section{Využitie v športe}

Systémy na získavanie polohy v priestore pomocou viacerých kamier sú dnes bežne
používané pri športoch. Systém Hawk Eye (stručne popísaný v
\citet*{owens2003hawk}) nielenže ponúka možnosť spracovanie súradníc, umožňuje
ale aj spätné prehranie situácie zo záznamu. Tento systém vyžaduje
vysokorýchlostné nákladné kamery a samotný softvér, ktorý nie je voľne
prístupný.

\section{Využitie v robotike}

Táto problematika je častokrát spomínaná aj so súvislosťou robotiky. Pre učely
súťaže RoboCup -- kategórie Soccer bolo vyvinutých niekoľko systémov pre
získavanie polohy lopty z obrazu kamery. Tieto roboty sú častokrát vybavené
všesmerovou kamerou (omnidirectional), ktorá poskytuje pohľad o zrkadlo a teda
zachytáva 360° obraz. Pre zlepšenie presnosti tohto systému bola všesmerová
kamera doplnená o obyčajnú kameru. Tento systém sa ale nachádza na pohyblivom
robotovi, narozdiel od nášho problému statických kamier.  Taktiež využíva
znalosť parametrov objektu. Hľadaným objektom je farebne odlišná lopta s vopred
známou veľkosťou. Tento systém je podrobnejšie popísaný v článku
\citet*{kappeler20103d}.

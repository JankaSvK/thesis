\chapter{Calibration}
\todo[inline]{intrintic parameters}
\todo[inline]{Obrazok sachovnice}
\todo[inline]{Ake minimalne rozmery sachovnica musi mat}
\todo[inline]{ze je lepsia asymetricka sachovnica, lebo ma len jednu osu a ze musia byt body najdene aj v spravnom poradi}

\todo[inline]{fancy odstavec o uvode}

Since we have no information about the camera and their positions, we use an
well known object to obtain these information.

\subsection*{Math notation}

A word \emph{vector} denotes a column vector in a shape of $n\times1$.

A matrix operation $W = (A|B)$, where $A$ is a matrix $m \times n$ and B is a
matrix $m \times p$, creates $W$, where first $n$ columns are the entries from
matrix $A$ and last columns consist of the columns of the matric $B$.
Example:

\[
A = \begin{pmatrix}
	1 & 2 \\
	3 & 4
\end{pmatrix},
B = \begin{pmatrix}
5 \\
6
\end{pmatrix},
(A|B) = \begin{pmatrix}
	1 & 2 & 5 \\
	3 & 4 & 6
\end{pmatrix}
\]

\subsubsection*{Homogenous coordinates}



\section{Intrinsic parameters}

Intrinsic camera parameters define a transform between a world coordinates and
the coordinates withing an image plane (in pixels). This transformation is
influenced by many factors, physical attributes of the camera.

These parameters include focal length, position of the principal point and
distortion coefficients. All these parameters are needed to get a correct
transformation between the point in the space and the point at the image plane.

OpenCV provides us routines to obtain these information about the camera. Since
for their description many great book were written we will only shotly describe
the information their provide.

\subsection{Camera matrix}
The focal length and the position of the principal point are linear parameters
of the camera.  The approach of the camera matrix which OpenCV use is to think
as matrix 3\times3 matrix in the following format:

...

This matrix provides us a way to convert from world coordinates to pixel
coordinates in the image.

\subsection{Distortion coefficients}

Cameras are equipted with the lenses to provide more light. Therefore the lens
cause a various distortions. A fish-eye lenses are known for their distortion.
Even webcamera lens have distortion, but not so visible as the camera with a
fish-eye lens. It is important to correct these distortions.

Two distortion types cause significant effect on the image. The first one is the
radial distortion, creating barrel effect and the second one tangential.

The radial distortion is caused by the lens itself. It can be usually described
by three parameters. Highly distorted images (like from fish-eye) often needs
more parameters.  Since our system use webcameras, we will use only three
parameters.

In the ideal camera the lens would be placed parralel to the chip. Since such a
precision while assembling process is not possible, tangential distortion
arrises. For this distiortion we use two parameters.

We describe both distortion effects by five parameters. More about the meaning
of the parameters could be found in \ref{learning-opencv}.

An effect of the distortion could be seen on the image \ref{fig:distortion}.
\todo[inline]{Distortion effect}

\section{Stereo calibration}

After perfoming a calibration of both cameras we also need information about
their relative position to each other. This position can by described by six
parameters, three as an angles around each axes to rotate and three for
translation vector.

Stereo calibration routine can perform also monocalibration for both cameras,
but we use a separate results for better precision.

It is important to note, that we obtain a rotation matrix and a translation
vector from stereo calibration between the coordinate systems of the cameras.

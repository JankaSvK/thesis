\chapter*{Conclusion}
\addcontentsline{toc}{chapter}{Conclusion}

This thesis proposed a system for object localization in the 3D space.

The primary goal of the thesis was to provide a step by step guide for the task
of the localization of the object in 3D. We achieved the goal by the describing
a whole problem and solving it step by step in this thesis.

The thesis started with an overview of the calibration process, explaining the
essential elements of the calibration. We presented a short introduction to
calibration routines used by the OpenCV and we also described the results
obtained by this process.

We implemented detection-based trackers such as \simback{}, \hsv{} and \patt{}. We
explored available trackers in the open-source libraries. These trackers from
the OpenCV and Dlib are sequence-based. We proposed several statistics for
comparison of the trackers. We tested all trackers on several setups, measuring their
speed and accuracy. We also explored their abilities to track multiple objects
and to recover from occlusion. We presented the results in 
well-organized tables and provided images for better understanding concepts.

As a next step, we explained how the projection matrices are computed. We
explained their importance in the projection of a point from 3D space to 2D
view of each camera. The chapter also contains a description of simple
triangulation method.  The method shows us how to obtain a position of the
object in 3D space if the projection matrices and the positions in the images
are available.

When all previous steps were covered, we tested the proposed system in several
environments and the settings. We studied the accuracy of the system by static
experiments and computing the estimated distances between the points and
comparing it to real values. In the end, we also provided experiments focusing
on overall experience when using the application.

The application can calibrate automatically using a chessboard pattern, track
one or more objects with a chosen tracker, display the results of the localization
live. Furthermore, it can work with the recordings instead of the live camera views.
The data from the calibration and localization are automatically saved.

There are several areas that can be explored in further work. While testing the
application, we noticed a noise in some specific axis. Additional work could
explore the cause of the noise and possible methods to eliminate it. We also
explored, that the precision for the objects further away from the camera is
lower compared to precision of the closer ones. The question remains, which
parameters of our setup influence this precision and how much. A suitable
extension to this project would be the use of more cameras and exploring if
this setup improves the precision.

In the end, we consider the application usable in practice. The results are not
precise enough to provide reliable results in a precision of millimeters, but
usable in many cases.

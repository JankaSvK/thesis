\chapter*{Conclusion}
\addcontentsline{toc}{chapter}{Conclusion}

This thesis proposed a system for object localization in the 3D space.

The primary goal of the thesis was to provide a step by step guide for the task
of the localization of the object in 3D. We achieved the goal by the describing
whole problem and solving it step by step in this thesis.

The thesis started with an overview of the calibration process, explaining the
essential elements of the calibration. We presented a short introduction to the
process provided by the OpenCV, and we also described the results obtained by
this process.

We implemented detection-based tracker as Simple Background, HSV and Pattern
Matching. We explored available trackers in the open-source libraries. These
trackers from the OpenCV and Dlib are sequence-based. We proposed several
statistics for trackers comparison. We tested all trackers on several setups,
measuring their speed and accuracy. Also, we explored their abilities to track
multiple objects and to recover from occlusion. We presented the results in the
few well-organized tables and provided images for better understanding
concepts.

As a next step, we explained how the projection matrices are computed. We
explained their importance in projection the point in the 3D space to each
camera. The chapter also contains a description of simple triangulation method.
The method shows us, how to obtain a position of the object in 3D space if the
projection matrices and the positions in the images are available.

When all steps were covered, we tested proposed system in several environments
and the settings. We studied the accuracy of the system by static experiments
and computing the estimated distances between the points and comparing it to
real values. In the end, we also provided experiments focusing on overall
experience when using the application.

The application can calibrate automatically on a chessboard, track one or more
objects with a chosen tracker, display results of the localization live. Also,
it can work with the recordings instead of the live camera views. The data from
the calibration and localization are automatically saved.

The thesis leaves many questions opened and not explored by the thesis. When
testing the whole application, we discovered noise in some specific axis. Would
be worth to explore, what is causing this noise. Also, we explored, that the
precision for the objects further away from the camera is decreasing. The
question remains, which parameters of our setup and how they influence this
precision. As nice extension to this project would be an idea to use more
cameras, and test if the precision improves as expected.

In the end, we consider an application as usable in practice. The results are
not precise enough to provide us reliable results in a precision of
millimeters, but it can provide a record of the trajectory.

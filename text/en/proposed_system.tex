\chapter{Proposed system}

To be able to locate an object with no previous knowledge of the object (such
as its size, color and so on), two views of the same object are needed. The
views can be obtained by one moving camera or by multiple cameras. In this
thesis, we take a closer look at the second approach.

We propose a system with two cameras. We connect cameras via USB to a computer.
Two cameras provide enough information for object localization and make the
project usable also on low-budget. The placement of the cameras is important,
but no precise alignment is required. Cameras should share a significant part
of the view.

\todo[inline]{fotka rozostavenia}

Our solution is able to estimate the positions of the cameras (relatively to
each other) using an automated calibration process. After calibration, the user
marks an object to be tracked in the view of each camera. The estimated
position of the cameras and the object is then displayed. We describe each step
of the process in the following chapters.

\section{Tools}

For this computer vision task, we use a \emph{OpenCV} library. OpenCV is open
source computer vision library with many algorithms implemented (for example
calibration, triangulation, and so on). We use trackers which are now available
only in \emph{contribute} version. For more information and examples of usage
visit their webpage\footnote{\url{https://opencv.org/}}.

To provide a comparison of trackers, we also include one tracker from
\emph{Dlib}\footnote{\url{http://dlib.net/}} library. It is also open source
and an interface for the Python is provided.  Since this library focus is on
machine learning and it does not contain so many functions for computer vision
as the OpenCV library yet

\section{Notations}

Following chapter describes some procedures using math notion. To avoid
ambiguity we list them here.

\subsubsection*{Vector}
A word \emph{vector} denotes a column vector in a shape of $n\times1$.

\subsubsection*{Block matrix}
A matrix operation $W = (A|B)$, where $A$ is a matrix $m \times n$ and B is a
matrix $m \times p$, creates $W$, where first $n$ columns are the entries from
matrix $A$ and last columns consist of the columns of the matric $B$.
Example:

\[
A = \begin{pmatrix}
        1 & 2 \\
        3 & 4
\end{pmatrix},
B = \begin{pmatrix}
5 \\
6
\end{pmatrix},
(A|B) = \begin{pmatrix}
        1 & 2 & 5 \\
        3 & 4 & 6
\end{pmatrix}
\]

\subsubsection*{Homogenous coordinates}

In computer vision often homogenous coordinates instead of Cartesian are used.
These coordinates have one element added. The meaning of this element is a
scale factor.  Therefore, one point could be described with in an infinit ways
of writing the coordinates. As the advantage, they can sipmplify many
operation. Often it is possible to solve a task as a matrix multiplication, if
the point is represented homogenous coordinates.

For homogenous coordinates in three dimensional space holds:
$$ (x, y, z, 1)^T = (sx, sy, sz, s)^T $$
Similartly it works in 2D.

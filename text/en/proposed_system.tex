\chapter{Proposed system}

To be able to locate an object with no previous information about it (such as
size, color), two views of the same object are needed. This views can be
achieved by one moving camera or by multiple cameras. In this thesis, we will
take a closer look at the second approach.

We propose a system with two cameras. We connect cameras via USB to a computer.
Two cameras provide enough information for object localization and make the
project usable also on low-budget. The camera's placement is important, but no
precise alignment is required. Cameras should share a significant part of the
view.

\todo[inline]{fotka rozostavenia}

Our solution will be able to estimate the positions of the cameras (relatively
to each other) using an automated calibration process. After calibration, the
user marks an object to be tracked from views of each camera. The estimated
position of the cameras and the object will be displayed. We describe each step
of the process in the following chapters.

\section{Tools}

For this computer vision task, we use a \emph{OpenCV} library. OpenCV is Open
Source Computer Vision Library with many algorithms implemented (for example
calibration, triangulation, and so on). We use trackers which are now available
only in \emph{contribute} version. For more information and examples of usage
visit their webpage\footnote{\url{https://opencv.org/}}.

To provide a comparison of trackers, we also include one tracker from
\emph{Dlib}\footnote{\url{http://dlib.net/}} library. It is also open source and an
interface for the Python is provided.  Since this library focus is on machine
learning and it does not contain so many functions for computer vision yet than
the OpenCV library. 

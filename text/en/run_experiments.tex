\chapter{How to run experiments}

In thesis, we mentioned many different experiments. In this chapter we provide
a description how to run them. Please, firstly check the dependencies and if
the application is running correctly (follow instructions in the Chapter \ref{ch:user-documentation}).

All experiments are located in the directory \texttt{program/experiments}. We
will use this directory as our root in the following text. Also, every script
should be executed by Python3.

\section{Sample calibration}

We provide several calibration scenarios to try out. A script named
\verb+calibrate.py+ is available in  the directory \verb+localization/+. The
script sets options to calibrate from different pairs of the video. To run
calibration on given set of videos, run \verb+calibrate.py {id}+, where
\verb+{id}+ can be 16, 38, 43, 63. This \verb+{id}+ represents the true
distance between the cameras. The estimated distance will be outputted in the
program.

If the calibration cannot succeed, check the number of images needed for
calibration in the \verb+Config.py+ (located in the \verb+program/program+)
and set it to smaller number.

\section{Sample localization}

In the same directory, \verb+localization/+ a script to run
localization on different scenarios is available. Run \verb+track.py {distance}{id}+, 
where \verb+{distance}+ represents the distance between the cameras
(available 16, 38, 43) and the \verb+{id}+ represents the id of the pair for the
videos (for all values of distance available scenarios 1 or 2).

The script also outputs on the first line the command which was called to run
the application.

\section{Static localization experiments}

Experiments providing the results for localization without tracking are only
executable under Linux based systems.

In the directory \verb+localization_static/+ a script is  available.
Run the script \verb+get_data_for_ladder.py id+. Allowed \verb+id+ is 1, 2, 3 or 4.

This script create results used to compute distances between the dots. To
compute the distances, a script \verb+compute_distances.py id+ is available. As
\verb+id+ may be used again 1, 2, 3 or 4. 

\section{Tracker experiments}

All tracker's experiments are located in the directory \verb+trackers/+. The
experiments can be started by \verb+trackers\_experiments.py+. First argument
of the script represents the scenario for the experiment.

\subsection*{Speed and accuracy}

\subsubsection*{\texttt{trackers{\_}experiments.py 1}}

This experiments runs all trackers at the each frame of the video. To compute
an inaccuracy, we use Simple background tracker as the representative tracker
(displayed with the red bounding box).

At the end, results for all trackers are displayed. The first columns specifies
the tracker, the second represents FPS (more described in the chapter Tracker)
and the third the inaccuracy.

\subsubsection*{\texttt{trackers{\_}experiments.py 2}}

The second scenario consists of the same experiment, with the orange cap on the
top of the robot, to provide results also for \hsv{} tracker.

\subsection*{Under occlusion}

\subsubsection*{\texttt{trackers{\_}experiments.py 3 {id}}}

To test a behavior of the trackers under occlusion, we provide an experiment 3.

Admissible \verb+{id}+ is from \verb+0+ to \verb+8+.

\subsubsection*{\texttt{trackers{\_}experiments.py 6}}

Tests all trackers under occlusion -- same as experiment 3, but it use all trackers.

\subsubsection*{\texttt{trackers{\_}experiments.py 7}}

Test with thin tunnel to test trackers under partial occlusion.

\subsubsection*{\texttt{trackers{\_}experiments.py 8}}

Test with wide tunnel to test tracker under full occlusion.

\subsection*{Tracking multiple objects}

\subsubsection*{\texttt{trackers{\_}experiments.py 4 {id}}}

The fourth experiment focuses on tracking multiple objects. Again, the results were
evaluated by the human. Admissible \verb+{id}+ is from \verb+0+ to \verb+8+.

\subsubsection*{\texttt{trackers{\_}experiments.py 5}}

The same experiment as previous, now with changed objects to be onecolor for
the \hsv{} tracker.



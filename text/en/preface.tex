\chapter*{Introduction}
\addcontentsline{toc}{chapter}{Introduction}

Living in the 21st century, we are witnessing a huge improvement of the
computers and their abilities. Since the beginning of the era of computers,
people have been creating machines and programs that solve problems more
efficiently than humans.

Starting from the beginning, each human has two eyes, which provide stereo
vision. Stereo vision allows us to estimate how far and where exactly an object
in 3D space is.

Our goal is to propose a system that will have similar capabilities using a
computer and two web cameras. The system can be used to monitor and track a
movement of objects. For example, it can provide a trajectory of the quadcopter
flight or another autonomous robot. These obtained data can be used for
additional analysis.

To achieve the above goal, we need to solve a problem of object detection,
tracking, and stereo vision by providing image streams from two cameras. We
focus on an overall process from the camera calibration, through object
detection in both cameras to an estimation of the object position in 3D space.

We set as our technical goals the following requirements:
\begin{itemize}
\item automatic calibration,
\item choice of a tracker,
\item easy way to add and test a new tracker,
\item reproducible run - working with videos, in addition to live stream,
\item saving results from calibration and localization for later use,
\item displaying results live.
\end{itemize}

The thesis starts with an overview of the work done in the area of computer
vision and object localization. Then the thesis covers three steps of the
object localization. The first step of the localization is to obtain
information about the cameras. The Calibration chapter covers the process of
obtaining parameters to describe the camera model by viewing a specific
pattern. Enough images of the pattern provide enough information to get the
parameters describing the camera. Furthermore, positions of the pattern, where
both cameras see the pattern, can be used for stereo calibration. The stereo
calibration provides information about the relative positions of cameras to
each other. We will later use this relative position and the camera parameters
in the localization process.

The second step is to track an object. Firstly, the user marks an object in
camera's views. Then the object position estimation is provided by a tracker
algorithm. Many different approaches may be used for tracking. We divide
between the tracking algorithms which work with a sequence of images
(sequence-based) and the algorithms working only with one image
(detection-based). We describe several tracking algorithms from both groups. At
the end of the chapter, we provide an empirical comparison of mentioned
trackers.

The chapter Localization covers the steps needed to estimate a position of the
object in three-dimensional space. At the beginning of the chapter, we choose
and describe a coordinate system, using results from stereo calibration. Then
we derive projection matrices, which transform from the world coordinates to
the image coordinates. As the next step, we introduce simple triangulation,
which estimates the position of the object in 3D by providing calibration
results, projection matrices and the coordinates of the object in both camera
images.

At the end of the thesis, we provide results of an evaluation of the proposed
system. In the first experiment, we omit a tracker, and we evaluate the
accuracy of calibration and localization. Second experiments present results
from the tests of the program as a whole, from calibration through tracking
until localization.

In the appendix, we include user documentation for running the program and
programmer documentation for a better understanding of the code.

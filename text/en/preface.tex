\chapter*{Introduction}
\addcontentsline{toc}{chapter}{Introduction}

Living in the 21st century, we are witnessing a huge improvement of the
computers and their abilities. Since the beginning of the era of computers,
people have been creating machines and programs that solve problems more
efficiently than human.

From the beginning, a human has two eyes. He is equipped with stereo vision.
Moreover, man can use it from the beginning of his life. Two eyes provide
stereo vision, which is useful to estimate how far and where exactly in 3D
space is an observed object.

Our goal is to propose a system that will be capable of similar results using a
computer and web cameras. The system can be used to monitor and track movement
of the objects. For example, it can provide a trajectory of the quadcopter
flight, or another autonomous robot. These obtained data can be used for
additional analysis.

To achieve the above goal, we need to solve a problem of object detection,
tracking, and stereo vision by providing an image stream from two cameras. We
focus on an overall process from the camera calibration, through object
detection in both cameras to estimation of the object position in 3D space.

This thesis will be divided by the three steps to achieve our goal. The first
step is to obtain information about the cameras. The Calibration chapter covers the
process of obtaining a parameters to describe camera model by viewing specific
pattern. Enough images of specific pattern provide enough information to get
the parameters describing a camera.  Furthermore, positions of the pattern
where both cameras see the pattern can be used for a stereo calibration. The
stereo calibration provide information about the relative position of the
cameras to each other. This relative position and the camera parameters we will
later use in the localization process.

The second step is to track an object. Firstly, user defines the object and
then the object position estimation is provided by tracker algorithm. Many
different approaches may be used. We differenciate between the tracking
algorithms which work with a sequence of the images (sequence-based) and the
algorithms working only with one image (detection-based). We provide several
tracking algorithms from both groups. At the end of the chapter, we provide a
comparison of mentioned trackers.

The chapter Localization covers a steps needed to estimate a position of the
object in three-dimensional space. At the beginning of the chapter, we choose
and describe coordinate system, using results from stereo calibration. Then we
derive projection matrices, which transforms from the world coordinates to the
image coordinates. As a next step we introduce simple triangulation, which
estimates the position of the object in 3D by providing calibration results,
projection matrices and the coordinates of the object in both camera images.

At the end of the thesis, we provide results of an evaluation of the proposed
system. In the first experiments, we ommit a trackers and we evaluate a
accuracy of the calibration and localization. Final experiments presents
results from the tests of the program as a whole, from calibration through
tracking until localization. 

In the appendix we include user documentation for running the program and
programmer documentation for better understanding of the code.

\todo[inline]{Mozno by tu mohol zazniet aspon nastrel analyzy}
\todo[inline]{Ako stepan by som vymenovala goals}

\todo[inline]{Chyba ti cela kapitola analysis -- Problem analysis}

\todo[inline]{pridat clanok a michala bartaka a obdrzalka do referencii}
\todo[inline]{pridat opencv dokumentaciu do zdrojov}
\todo[inline]{Zadanie}
